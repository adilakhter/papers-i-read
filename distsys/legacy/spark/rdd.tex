\documentclass[12pt,a4paper,oneside]{article}
\usepackage[a4paper]{geometry}
\setlength\parindent{0pt}
\usepackage{hyperref}

\begin{document}

\begin{itemize}
  \item Name: Pinglei Guo
  \item Assignment: RDD
  \item Last Modified: \today
\end{itemize}

1. What is the problem authors are trying to solve

\medskip

Speed up large scale data analysis and allow expressing complex model using combination
of simple operations. Lower bandwith and keep data in memory.

\bigskip

2. How does the authors’ approach or solution improve on previous approaches to that problem

\medskip

\begin{itemize}
  \item Found the core problem instead just focus on the obvious problem. 'The common problem was a lack of data sharing abstractions'. And
  they introduced RDD
  \item The operations on RDD is more coarse grained
  \item Use Lineage for recovery, recompute the lost partition from parent RDD, or use chekpoint if the conpute chain is too long
\end{itemize}

\bigskip

3. Why is this work important

\medskip

Because Spark is much faster than Hadoop, meaning using less time and lower network traffic for same amount of data.
And RDD provide an high level abstraction that can be fit into many types of computing. Also Scala is not that verbose
as Java.

\bigskip

4. Your comments/questions

\medskip

Questions

\begin{itemize}
  \item They said they use original scala, but there are optimization to the shell, so I think they built a new shell based on Scala's REPL,
  and you can't import spark in normal Scala shell and run queries intereactively.
  \item Spark said they don't use swap when compare itself to other distributed memory systems, and later they said they use LRU when RAM is
  not enough, I think they should make it more explicit when they show their advantages with previous systems
  \item They mentioned they will try SQL as a new interface for running query, and the have Spark QL (which is next reading).
  But I thinking using general purpose programming language to express query is more concise though a little bit verbose. It seems everything about data endup providing SQL support.
  \item Is it possible to do some optimization if it has some knowledge of the underlying dataset, the paper only mentioned lazy evaluation and
  partition based on locality. Also in some environment using Docker, like Kubernetes, does the partition logic still work? Since most processes may not be long running.
  \item It would cost a lot to run their benchmark using AWS, pretty envy about that.
  \item Some lower level things like the fault tolerance of driver and worker as not mentioned, they just said they use
  Mesos, though Mesos is also from AMP lab.
\end{itemize}

\end{document}
