\documentclass[12pt,a4paper,oneside]{article}
\usepackage[a4paper,left=3cm,right=2cm,top=2.5cm,bottom=2.5cm]{geometry}
\usepackage[utf8]{inputenc}
\usepackage[backend=biber,style=numeric,sorting=none]{biblatex}
\usepackage{hyperref}
% For code highlight
\usepackage{listings}
% FIXME: the font is really strange
\usepackage[T1]{fontenc}

\addbibresource{summary.bib}

% What problem does the paper solve? Is is important?
% How does it solve the problem?
% What alternative approaches are there? Are they adequately discussed in the reading?
% How does this work relate to other research, whether covered in this course or not?
% What specific research questions, if any, does the paper raise for you?

\title{Summary for: Time, Clocks, and the Ordering of Events in a Distributed System}
\author{Pinglei Guo}
\date{\today}

\begin{document}

\maketitle

\section{Paper walk through}

First introduce the order, what is `happen after` in distributed system, which
leads to logic clock

\section{Simple implementation}

Some simple C code could do that

You can write the code
% e ... the code highlight sucks ....
\begin{lstlisting}[language=Python]
import numpy as np

def incmatrix(genl1,genl2):
    m = len(genl1)
    n = len(genl2)
    M = None #to become the incidence matrix
    VT = np.zeros((n*m,1), int)  #dummy variable

    #compute the bitwise xor matrix
    M1 = bitxormatrix(genl1)
    M2 = np.triu(bitxormatrix(genl2),1)

    for i in range(m-1):
        for j in range(i+1, m):
            [r,c] = np.where(M2 == M1[i,j])
            for k in range(len(r)):
                VT[(i)*n + r[k]] = 1;
                VT[(i)*n + c[k]] = 1;
                VT[(j)*n + r[k]] = 1;
                VT[(j)*n + c[k]] = 1;

                if M is None:
                    M = np.copy(VT)
                else:
                    M = np.concatenate((M, VT), 1)

                VT = np.zeros((n*m,1), int)

    return M
\end{lstlisting}

or include a file (Chinese is not supported .... don't need to figure it out now, so just leave it)

\lstinputlisting[
  language = C
]{demo.c}


\section{Discussion}

\subsection{Problem solved}

The paper solve the problem of total ordering in distributed system.
It is, because it can solve many synchronization problems.

By introducing logical clocks and a method to adjust physical clocks.

\subsection{Alternative}

No, no alternative for logical clocks. Oops I guess there are, things
like vector clock and stuff may work LOL. believe it or not.

\section{Related research}

You can find more in \ref{l1}
You can also use page \pageref{l1}

\section{Q\&A}

The example in this paper use a lot of assumptions to simplify the proof.
In real world applications, how engineers deal with these assumptions like messages
arrive in order.


You can jump to me

I want to cite\cite{lamport1978time}

\section{Supplemental Materials}

\subsection{Terms} \label{l1}

\subsection{Video and Slides}

\printbibliography

\end{document}

This is never printed
