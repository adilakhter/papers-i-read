\documentclass[12pt,a4paper,oneside]{article}
\usepackage[a4paper]{geometry}
\setlength\parindent{0pt}
\usepackage{hyperref}

\begin{document}

\begin{itemize}
  \item Name: Pinglei Guo
  \item Assignment: The log structured merge tree
  \item Last Modified: \today
\end{itemize}

1. What is the problem authors are trying to solve

\medskip

Heavy write load and acceptable read spead

\bigskip

2. How does the authors’ approach or solution improve on previous approaches to that problem

\medskip

\begin{itemize}
  \item Increase write proformance by write to memory
  \item Increase disk write speed by append only and batching
\end{itemize}

\bigskip

3. Why is this work important

\medskip

Because write heavy is become more common when we have more user, more devices, more data

\bigskip

4. Your comments/questions

\medskip

Questions

\begin{itemize}
  \item Do we still need WAL when we use log structured merge tree
  \item Is tree the core concept for LSM tree? No, I think merge is
  \item A lot of databases using LSM Tree are column family databases, are they similar to column store?
  \item What is the proper level depth and size of each level
  \item How to speed up read when you need to read and merge result from multiple levels
  \item How to use Bloomfilter for range queries? Use prefix can solve some (but not all)
\end{itemize}

\end{document}
