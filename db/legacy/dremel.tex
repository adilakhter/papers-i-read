\documentclass[12pt,a4paper,oneside]{article}
\usepackage[a4paper]{geometry}
\setlength\parindent{0pt}
\usepackage{hyperref}

\begin{document}

\begin{itemize}
  \item Name: Pinglei Guo
  \item Assignment: Dremel: Interactive Analysis of Web-Scale Datasets
  \item Last Modified: \today
\end{itemize}

1. What is the problem authors are trying to solve

\medskip

Storing nested data model in columnar store for fast OLAP query

\bigskip

2. How does the authors’ approach or solution improve on previous approaches to that problem

\medskip

\begin{itemize}
  \item columnar format for nested data model (previously, they only apply to flat)
  \item faster than MR, easy to play with data instead of launch a MR and regret
  \item in situ, access data in place (locality?)
  \item execute directly, not a DSL wrapper for MR jobs like Pig
\end{itemize}

\bigskip

3. Why is this work important

\medskip

Because columnar store are well known for OLAP, but previously people have only used in on flat row model (in traditional RDBMS),
they managed to apply it to nested data model (document model?) and developed the algorithm to stripe and reconsturct the record

\bigskip

4. Your comments/questions

\medskip

Questions

\begin{itemize}
  \item How does Dremel work with other row based systems like Bigtable (column family is not columnar store I think), F1 in google.
  Do we need to store an extra copy of data in columnar form for fast OLAP query
  \item Spark SQL also mentioned (in memory) columnar store I recall (but not clearly), it seems to be difference from Dremel's
  \item Compression, which is used a lot columnar store is not mentioned (or I missed it), like run length
  \item 'Dremel’s codebase is dense; it comprises less than 100K lines of C++, Java, and Python code.' so 100K is not a big number for software at this size I guess?
  \item There is an opensource implementation: Apache Parquet \url{https://parquet.apache.org/}
\end{itemize}

\end{document}
