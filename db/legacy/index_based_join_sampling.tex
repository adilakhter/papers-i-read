\documentclass[12pt,a4paper,oneside]{article}
\usepackage[a4paper]{geometry}
\setlength\parindent{0pt}

\begin{document}

\begin{itemize}
  \item Name: Pinglei Guo
  \item Assignment: Cardinality Estimation Done Right: Index-Based Join Sampling
  \item Last Modified: \today
\end{itemize}

1. What is the problem authors are trying to solve

\medskip

Estimate cardinality in an accurate and efficient manner to optmize execution plan in database

\bigskip

2. How does the authors’ approach or solution improve on previous approaches to that problem

\medskip

\begin{itemize}
  \item consider modern hardware, where memory is cheaper and a lot of data sits in memory instead page in and out with disk regularly
  \item utilizes existing index structure, which is 'much cheaper than joining independent samples as the sample size stays roughly constant after each additional join'
  \item have a timeout, fallback to traditional if sampling took too long
\end{itemize}

\bigskip

3. Why is this work important

\medskip

\begin{itemize}
  \item join is an important features for RDBMS, and is also very useful for OLAP, with data size grown, clever execution plan
  is needed to keep response time at an accepted level. (a.k.a user can write stupid query but database should execute them in a smart way)
  \item the sampling overhead is acceptable and has fallback
\end{itemize}

\bigskip

4. Your comments/questions

\medskip

Questions

\begin{itemize}
  \item SparkQL also mentioned their optimization, can this work be applied to Spark? I think they didn't mention join much in the paper,
  btw: most time Spark use other RDBMS as data source via JDBC, it does not have much control.
  \item Materialized is pretty popular these days, which you put everything together when you create the new record, i.e. Cassandra,
  which use space to trade time. Is it possible to make the database to be smart enough to decide when it should materialize, when it
  should keep the relation, and make this operation transparent to the user.
\end{itemize}

\end{document}
