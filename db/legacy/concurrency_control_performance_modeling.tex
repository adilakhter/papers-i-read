\documentclass[12pt,a4paper,oneside]{article}
\usepackage[a4paper]{geometry}
\setlength\parindent{0pt}
\usepackage{hyperref}

\begin{document}

\begin{itemize}
  \item Name: Pinglei Guo
  \item Assignment: Concurrency Control Performance Modeling: Alternatives and Implications
  \item Last Modified: \today
\end{itemize}

1. What is the problem authors are trying to solve

\medskip

Survey on existing concurrency control model, and explain previous benchmarks

\bigskip

2. How does the authors’ approach or solution improve on previous approaches to that problem

\medskip

\begin{itemize}
  \item clarify the assumption, which is why previous benchmarks has seems contradictory result
  \item benchmarks under different assumptions
\end{itemize}

\bigskip

3. Why is this work important

\medskip

\begin{itemize}
  \item It points out, although previous benchmarks seems contradictory, it is a result of different assumption.
  While they are all correct under their own assumption, they could be wrong when placed under other benchmarks assumption.
  \item When there is limited resource, reserve (blocking) until have to restart is better than restart transaction right away.
\end{itemize}

\bigskip

4. Your comments/questions

\medskip

Questions

\begin{itemize}
  \item The assumption part is really important, some time series databases benchmarks I saw simplying gave out throughput and response time
without mentioning what their worload is, how many hardware resources, even changing a parameter in linux config can result in great
performance different, and sometimes people just focus on the final result instead of asking how the numbers are obtained.
  \item Throughput is their main metrics, however in today's world, a lot interactive query, response time is also a major
metric, in both OLAP and/or OLTP systems
  \item At the end of paper, they quote and say we would always to have to deal with finite resources even though the price of resources are decreasing.
I don't think this is the case when we go beyond the scope of earth (in sci-fi movies), data transmisson in space means extremely high delay (speed of light),
and in that case, compute resource in actually infinite.
I think physical limit will become the real limit for a lot of computer science problems at last.
\end{itemize}

\end{document}
