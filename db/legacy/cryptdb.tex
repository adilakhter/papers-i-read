\documentclass[12pt,a4paper,oneside]{article}
\usepackage[a4paper]{geometry}
\setlength\parindent{0pt}
\usepackage{hyperref}

\begin{document}

\begin{itemize}
  \item Name: Pinglei Guo
  \item Assignment: CryptDB: protecting confidentiality with encrypted query processing
  \item Last Modified: \today
\end{itemize}

1. What is the problem authors are trying to solve

\medskip

Store encrypted data in DBMS, so people with full access to the database (i.e. DBA, Hacker)
can't get sensetive information (except those currently logged in)

\bigskip

2. How does the authors’ approach or solution improve on previous approaches to that problem

\medskip

\begin{itemize}
  \item Put the encryption in the server side, no special requirement in client side
  \item Combines the encryption logic with SQL (more application specific encryption)
  \item Allows advanced ACL, like visible inside a group via Key Chaining
\end{itemize}

\bigskip

3. Why is this work important

\medskip

It provides an almost drop in solution for popular RDBMS for encrypting sensitive
information.

\bigskip

4. Your comments/questions

\medskip

Questions

\begin{itemize}
  \item What happens whe user change password, group (in phpBB forum) change access?
  \item The overhead (20\%) is small for applications like phpBB, but what about facebook, even 1\% means a lot of money
  \item RDBMS are not the only datastore, also even for RDDMS, the full text search functionality is often provided by external application
  like Solr and Elasticsearch, can the proxy work with NoSQL, search engine
  \item Is there any production environment using CryptoDB like encryption proxy method? The paper has been there for a while (since 2011)
  \item How do large companies avoid their engineers from accessing sensitive user data
  \item When there are data mining and machine learning stuff, CryptoDB can't avoid exposing sensitive personal information.
  i.e. sometimes even the proccessed data can be inversed and point the exact person.
\end{itemize}

\end{document}
