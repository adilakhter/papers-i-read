\documentclass[12pt,a4paper,oneside]{article}
\usepackage[a4paper]{geometry}
\setlength\parindent{0pt}

\begin{document}

\begin{itemize}
  \item Name: Pinglei Guo
  \item Assignment: Bigtable
  \item Last Modified: \today
\end{itemize}

1. What is the problem authors are trying to solve

\medskip

Google has (real) big data and various workload.
A lot them does not require the ACID features of traditional RDBMS.
They need a system to support large volumne of data and is reliable

\bigskip

2. How does the authors’ approach or solution improve on previous approaches to that problem

\medskip

They use a LSM tree structure to improve write performance, and use GFS for reliable storage.
The column faimly model built on top of key value leads to more possible optimization (i.e. locality)
and more high level interface for developers (than just K-V)

\bigskip

3. Why is this work important

\medskip

\begin{itemize}
  \item the paper let more people know LSM tree
  \item It shows how you can build a structural data store on top of distributed file system
  \item It works for Google
\end{itemize}

\bigskip

4. Your comments/questions

\medskip

Questions

\begin{itemize}
  \item Maybe using a non distributed file system, and let the database take control how the SSTable files
are replicated could gain better performance, i.e. the logic for file replication could be more application
specific and enable more optimization
  \item Would the need for master and Chubby become the bottle neck of the system?
  \item Having multiple tablet is like the partition by primary key in some RDBMS like MySQL, so the
  \item The idea of master kill itself if it can't reach Chubby is pretty interesting, never thought
  about having master terminate itself when it found itself is partitioned from rest of the network
  (or just the coordiante service)
\end{itemize}

Miscellaneous

\begin{itemize}
  \item Bigtable provide security features like ACL on column faimly level
  \item support 'get by key range' operation, which is pretty useful when it comes to things like timestamp
\end{itemize}

\end{document}
