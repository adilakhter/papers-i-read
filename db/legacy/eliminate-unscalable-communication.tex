\documentclass[12pt,a4paper,oneside]{article}
\usepackage[a4paper]{geometry}
\setlength\parindent{0pt}
\usepackage{hyperref}

\begin{document}

\begin{itemize}
  \item Name: Pinglei Guo
  \item Assignment: Eliminating unscalable communication in transaction processing
  \item Last Modified: \today
\end{itemize}

1. What is the problem authors are trying to solve

\medskip

Concurrent access to a shared resource is the bottleneck of many databases, especially
shared everything type. And some downgrade is unacceptable for certain use cases like bank.
The author wants to increase performance while keep the same level of functionality

\bigskip

2. How does the authors’ approach or solution improve on previous approaches to that problem

\medskip

\begin{itemize}
  \item It focus on communication pattern and break it into three types, and downgrade the most
  cost one (unbounded) to others
  \item It doesn't not sacrifices ACID
  \item Previous approaches just focus on the most obvious critical path regardless of their type,
  while in the long term eliminating unbounded communication is the key
\end{itemize}

\bigskip

3. Why is this work important

\medskip

Because it can let databases make use of modern multicore hardware and adapt to modern workload.
And spend more on real logical instead of waiting for (communicate with) each other

\bigskip

4. Your comments/questions

\medskip

Questions

\begin{itemize}
  \item This paper feels like a summary (survey) of a bunch of related work the author have done.
  \item The idea of this paper is pretty similar to CPU piplining, break big transaction into small ones for parallel,
  pipelining etc.
  \item It mentiond logical partition and physical partition, but I think a lot of application and databases now
  focusing on partition on bussiness level, when you design the schema, you make sure they are splited into several nodes
  instead of all write to one node.
  \item The idea of letting requests communicate with each other to eliminate themselves or pass lock among themselves without
  hitting the central resource manager is great, but what if some requests are malformed (i.e. in a large deployment,
  some nodes are running an older version of software and the protocol has changed in newer version but they can still communicate
  except they misunderstand each other), and without a central resource manager sometimes it is hard to log and track when and why bad things happend.
\end{itemize}

Stuff

\begin{itemize}
  \item  A lot of Open Source databases are mentioned in Section 2 are mentioned that don't scale, and the author then said
  some of them have improved themselves based on the author's papers
\end{itemize}

\end{document}
