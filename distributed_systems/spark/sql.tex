\documentclass[12pt,a4paper,oneside]{article}
\usepackage[a4paper]{geometry}
\setlength\parindent{0pt}
\usepackage{hyperref}

\begin{document}

\begin{itemize}
  \item Name: Pinglei Guo
  \item Assignment: Spark SQL
  \item Last Modified: \today
\end{itemize}

1. What is the problem authors are trying to solve

\medskip

Make Spark's query interface more expressive and reduce the learning curve to many (database) users.
Query from heterogeneous sources

\bigskip

2. How does the authors’ approach or solution improve on previous approaches to that problem

\medskip

\begin{itemize}
   \item Not limited by legacy problem of Hive like Shark, which is hard to be used on relational data source
   \item Use the features of Scala to make the query interface more programmable than simple query language
   \item Introduce an extensible optimizer which requires few code to add new rules
\end{itemize}

\bigskip

3. Why is this work important

\medskip

Spark is used in a lot of areas, from BI to machine learning and include multiple datasource from JSON file to
RDBMS, NoSQL store. Spark SQL provides a unified programmable query interface to make Spark more easy to use.

\bigskip

4. Your comments/questions

\medskip

Questions

\begin{itemize}
  \item A lot of features in the optimizer are implemented using Scala's own features like pattern matching, byte code generation.
  Is it possible to use a non functional language like C++ to do the same, and how much is the extra effort and performance gain
  \item Spark SQL is more like LINQ than SQL, from the example, you can actually building an AST when you create a query, so the
  SQL layer actually don't need parser, but the plain text representation layer also has its advantage, like it is easier to write
  SQL in normal web admin application.
  \item The cost model is not in very detail, what data is needed from datasource in order to have Spark SQL make better plan, and
  if there is several levels of optimization based on the knowledge of the datasource
  \item The dataframe idea is borrowed from R but use lazy evaluation
\end{itemize}

\end{document}
