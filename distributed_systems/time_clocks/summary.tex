\documentclass[12pt,a4paper,oneside]{article}
\usepackage[a4paper,left=3cm,right=2cm,top=2.5cm,bottom=2.5cm]{geometry}
\usepackage[utf8]{inputenc}
\usepackage[backend=biber,style=numeric,sorting=none]{biblatex}
\usepackage{hyperref}
\usepackage{amsmath}
\usepackage{centernot}
% For code highlight
\usepackage{listings}
% FIXME: the font is really strange
\usepackage[T1]{fontenc}

\addbibresource{summary.bib}

% What problem does the paper solve? Is is important?
% How does it solve the problem?
% What alternative approaches are there? Are they adequately discussed in the reading?
% How does this work relate to other research, whether covered in this course or not?
% What specific research questions, if any, does the paper raise for you?

\title{Summary for: Time, Clocks, and the Ordering of Events in a Distributed System}
\author{Pinglei Guo}
\date{\today}

\begin{document}

\maketitle

\begin{abstract}

This is a paper summary for Time, Clocks, and the Ordering of Events in a Distributed System\cite{l} by Lamport\footnote{\url{https://en.wikipedia.org/wiki/Leslie_Lamport}}.
It is the assignment of CMPS232-Fall16\footnote{\url{https://github.com/palvaro/CMPS232-Fall16}}.
% TODO: better 1-2 line summary
In this paper Lamport introduced logical clocks for ordering event totally thus solving synchronization problems like mutual exclusion.
It is widely used in both academy and industry.

\end{abstract}

\section{Paper walk through}

This paper mainly covers the following part.

\begin{enumerate}
  \item Introduction
  \begin{enumerate}
    \item Definition of Distributed Systems
  \end{enumerate}
  \item The Partial Ordering
  \begin{enumerate}
      \item Definition of order, what is \textit{happens before}, Symbol $\rightarrow$. % the arrow http://garsia.math.yorku.ca/MPWP/LATEXmath/node9.html
      \item space-time diagram
  \end{enumerate}
  \item Logical Clocks
  \begin{enumerate}
      \item 2 Clock Condition, one inside process (C1), one between process, measured by message (C2).
      \item improved and 3-D space-time diagram.
      \item 2 Implementation Rule, one inside process (IR1), one between process , measured by message, $T_m = C_i\langle a \rangle$ (IR2).
  \end{enumerate}
  \item Ordering the Events Totally
  \begin{enumerate}
      \item Definition of total ordering using Logical Clock and order among processes, Symbol $\Rightarrow$
      \item Solve mutual exclusion problem, fixed number of processes share a single resource.
      \item Specify the synchronization as a State Machine.
      \item Pitfall, handling failure (related paper is referred).
  \end{enumerate}
  \item Anomalous Behavior
  \begin{enumerate}
      \item Precedence information based on messages external to the system cause anomalous behavior.
      \item Solution 1, \textit{explicitly introduce info about the ordering $\rightarrow$ to the system} by user.
      \item Solution 2, Strong Clock Condition, thus use Physical Clocks to eliminate anomalous behavior.
  \end{enumerate}
  \item Physical Clocks
  % TODO: this part is a lot harder, and I skipped the Proof of the Theorem in appendix /w\
  \begin{enumerate}
      \item Requirements for the clock, individual (PC1), between clocks, small variation (PC2).
      \item Requirements for preventing anomalous behavior.
      \item Algorithm to ensure PC2, clock synchronization.
      \item The length of time for synchronizing clocks.
  \end{enumerate}
\end{enumerate}

Following is the reading notes for each section

\subsection{Introduction}

\begin{quote}
  A distributed system consists of a collection of distinct processes which are
spatially separated, and which communicate with one another by exchanging messages. \\
  A system is distributed if the message transmission delay is not negligible compared
to the time between events in a single process.\cite{l} \\

% TODO: right align and add a horizontal line
\textit{Leslie Lamport}
\end{quote}

The order of event is simple if we only consider physical time, like I woke on 8:00AM happens before
I eat my breakfast on 8:30AM.
But communication takes time due to spatial separation or other facts.

% TODO: a real word example
% FIXME: this example still feels quite wired
I woke up on 8:00AM, send a message to Jack on 8:05AM and ate breakfast on 8:10AM, Jack saw the message
on 8:30AM.
So although in physical time breakfast \textit{happens before} Jack saw the message, they are \textit{concurrent}
in the view of the system.

\subsection{The Partial Ordering} \label{PO}

Order need to be observed with the system.
Physical clocks are not perfectly accurate and do not keep precise physical time\cite{l}.

Define \textit{happens before} ($\rightarrow$) as following

\begin{enumerate}
  % TODO: may use the word in paper directly for PO1, due to C1
  \item \label{PO1} Inside a process, $\rightarrow$ is just like physical time.
  \item \label{PO2} If message is sent from process $P_i$ to $P_j$,
  $a$ is the sending of the message,  $b$ is receiving of that message, then $a \rightarrow b$.
  \item $if\ a \rightarrow b\ and\ b \rightarrow c, then\ a \rightarrow c$.
  % TODO: IMPL the counter implementation
  \item $if\ a \centernot\rightarrow b\ and\ b \centernot\rightarrow a$, then a is \textit{concurrent} with b.
\end{enumerate}

\subsection{Logical Clocks}

Define \textit{Clock} $C$ as following\cite{l}

\begin{itemize}
  \item just a way of assigning numbers to event, no relation with physical time, can be implemented using mere counter.
  \item $C_i$ for each process $P_i$ where a number $C_i\langle a \rangle$ is assigned to event $a$ in that process.
  \item function $C$ represents the entire system of clocks, so $C\langle b \rangle = C_i \langle b \rangle$ if $b$ is an event in process $P_i$.
\end{itemize}

Define \textit{Clock Condition} as following

\medskip

For any events $a, b$: if $a \rightarrow b$ then $C\langle a \rangle < C\langle b \rangle$

\medskip

By extending \ref{PO1} and \ref{PO2} in \ref{PO} we have

\begin{enumerate}
  \item \label{C1} C1. If $a$ and $b$ are events in process $P_i$, and $a$ comes before $b$, then $C\langle a \rangle < C\langle b \rangle$\cite{l}
  \item \label{C2} C2. If message is sent from process $a$ to $b$, then $C\langle a \rangle < C\langle b \rangle$\
\end{enumerate}

By extending \ref{C1} and \ref{C2} we have the implementation rule

% TODO: IMPL 
\begin{enumerate}
  \item \label{IR1} IR1. Each process $P_i$ increments $C_i$ between any two successive events\cite{l}.
  \item \label{IR2} IR2.
  \begin{enumerate}
    \item If event $a$ is the sending of a message $m$ by process $P_i$, then $m$ contains a timestamp $T_m=C_i\langle a \rangle$\cite{l}.
    \item Upon receiving a message $m$, process $P_j$ sets $C_j$ greater than or equal to its present value and greater than $T_m$\cite{l}.
  \end{enumerate}
\end{enumerate}

\subsection{Ordering the Event Totally}

\subsection{Anomalous Behavior}

\subsection{Physical Clocks}

\section{Simple implementation}

Some simple C code could do that

\begin{lstlisting}[language=C]
#include <stdio>

int main(char* args){
  print("Hello Clock!")
}
\end{lstlisting}

\section{Discussion}

\subsection{Problem solved}

It solved the following problems

determine order partially and totally. % TODO: why is this important

The paper solve the problem of total ordering in distributed system.
It is, because it can solve many synchronization problems.

By introducing logical clocks and a method to adjust physical clocks.

\subsection{Alternative}

No, no alternative for logical clocks. Oops I guess there are, things
like vector clock and stuff may work LOL. believe it or not.

\section{Related research}


Too many related researches

You can find more in \ref{l1}
You can also use page \pageref{l1}

\section{Q\&A}

% TODO: got this question when writing the introduction part
Q: When deal with message, the 'time' of send and receive message is regarded as same timestamp? \\
A: The clock must ticks between the message,  % TODO: find the pos in paper


Q: How visualize the space-time diagram to 3D \\
A: IDK

The example in this paper use a lot of assumptions to simplify the proof.
In real world applications, how engineers deal with these assumptions like messages
arrive in order.

I want to cite\cite{l}

\section{Supplemental Materials}

\subsection{Terms} \label{l1}

Since most terms I learned about math and computer science was in Chinese. I found a lot of terms blocking
me from understanding this paper, so are listed them below for fellow students whose mother language is not English.

\medskip

% TODO: may use subsubsection

Converse, Inverse, Contrapositive\footnote{\url{http://hotmath.com/hotmath_help/topics/converse-inverse-contrapositive.html}}.
Given a statement $if\ p, then\ q$, $p$ is called hypothesis, $q$ is called $conclusion$.

\begin{itemize}
    \item Converse  $if\ q, then\ p$
    \item Inverse $if\ not\ p, then\ not\ q$
    \item Contrapositive $if\ not q, then\ not\ p$
\end{itemize}

If the statement is true, then the contrapositive is also true.
And the inverse is the contrapositive for the converse, so if the inverse is true, the converse is true.

\medskip

% FIXME: this website has annoying ads, why its the first result in google
Mutual Exclusion\footnote{\url{https://www.techopedia.com/definition/25629/mutual-exclusion-mutex}}
also known as mutex, which you may have heard it when reading books about multithread.
It is \textit{a program object that prevents simultaneous access to a shared resource}.

\subsection{Video and Slides}

TBD: from https://github.com/at15/CMPS232/issues/6

\section{Changelog}

\begin{itemize}
  \item 09/25/2016 \ Initial version
\end{itemize}

\printbibliography

\end{document}

This is never printed
