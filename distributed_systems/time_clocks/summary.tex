\documentclass[12pt,a4paper,oneside]{article}
\usepackage[a4paper,left=3cm,right=2cm,top=2.5cm,bottom=2.5cm]{geometry}
\usepackage[utf8]{inputenc}
\usepackage[backend=biber,style=numeric,sorting=none]{biblatex}
\usepackage{hyperref}
% For code highlight
\usepackage{listings}
% FIXME: the font is really strange
\usepackage[T1]{fontenc}

\addbibresource{summary.bib}

% What problem does the paper solve? Is is important?
% How does it solve the problem?
% What alternative approaches are there? Are they adequately discussed in the reading?
% How does this work relate to other research, whether covered in this course or not?
% What specific research questions, if any, does the paper raise for you?

\title{Summary for: Time, Clocks, and the Ordering of Events in a Distributed System}
\author{Pinglei Guo}
\date{\today}

\begin{document}

\maketitle

\section{Paper walk through}

let's walk it through

\section{Simple implementation}

Some simple C code could do that

\begin{lstlisting}[language=C]
#include <stdio>

int main(char* args){
  print("Hello world?")
}
\end{lstlisting}

\section{Discussion}

\subsection{Problem solved}

The paper solve the problem of total ordering in distributed system.
It is, because it can solve many synchronization problems.

By introducing logical clocks and a method to adjust physical clocks.

\subsection{Alternative}

No, no alternative for logical clocks. Oops I guess there are, things
like vector clock and stuff may work LOL. believe it or not.

\section{Related research}


Too many related researches

You can find more in \ref{l1}
You can also use page \pageref{l1}

\section{Q\&A}

The example in this paper use a lot of assumptions to simplify the proof.
In real world applications, how engineers deal with these assumptions like messages
arrive in order.

I want to cite\cite{lamport1978time}

\section{Supplemental Materials}

\subsection{Terms} \label{l1}

Since all the terms I learned about math was in Chinese. I need to review some basic
terms when it comes to proof etc.

\medskip

Converse, Inverse, Contrapositive\footnote{\url{http://hotmath.com/hotmath_help/topics/converse-inverse-contrapositive.html}}.
Given a statement $if\ p, then\ q$, $p$ is called hypothesis, $q$ is called $conclusion$.

\begin{itemize}
    \item Converse  $if\ q, then\ p$
    \item Inverse $if\ not\ p, then\ not\ q$
    \item Contrapositive $if\ not q, then\ not\ p$
\end{itemize}

if and only if

mutex, mutual exclusion

\subsection{Video and Slides}

TBD: from https://github.com/at15/CMPS232/issues/6

\printbibliography

\end{document}

This is never printed
