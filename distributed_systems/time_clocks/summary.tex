\documentclass[12pt,a4paper,oneside]{article}
\usepackage[a4paper,left=3cm,right=2cm,top=2.5cm,bottom=2.5cm]{geometry}
\usepackage[utf8]{inputenc}
\usepackage[backend=biber,style=numeric,sorting=none]{biblatex}
\usepackage{hyperref} % for auto jump
\usepackage{amsmath}
\usepackage{centernot} % for \ on -> \centernot\rightarrow generates does not imply
% For code highlight
\usepackage{listings}
% FIXME: the font is really strange
\usepackage[T1]{fontenc}
\setlength\parindent{0pt}
\addbibresource{summary.bib}

\title{Summary for: Time, Clocks, and the Ordering of Events in a Distributed System}
\author{Pinglei Guo}
\date{\today}

\begin{document}

\maketitle

\begin{abstract}

\textbf{WIP}\footnote{The latest can be found in \url{https://github.com/at15/papers-i-read/tree/master/distributed_systems/time_clocks}}.
This is a paper summary for Time, Clocks, and the Ordering of Events in a Distributed System\cite{l} by Lamport\footnote{\url{https://en.wikipedia.org/wiki/Leslie_Lamport}}.
It is the assignment of CMPS232-Fall16\footnote{\url{https://github.com/palvaro/CMPS232-Fall16}}.
% TODO: the summary is not accurate
In this paper Lamport introduced logical clocks and total ordering with examples of solving synchronization problems like mutual exclusion, physical clocks.
It is widely used in both academy and industry.

\end{abstract}

\section{Paper walk through}

This paper mainly covers the following part.

\begin{enumerate}
  \item Introduction
  \begin{enumerate}
    \item \ref{s:intro:s:dist} Definition of Distributed Systems
  \end{enumerate}
  \item The Partial Ordering
  \begin{enumerate}
      \item \ref{s:PO:DO} Definition of order, what is \textit{happens before}, Symbol $\rightarrow$. % the arrow http://garsia.math.yorku.ca/MPWP/LATEXmath/node9.html
      \item space-time diagram
  \end{enumerate}
  \item Logical Clocks
  \begin{enumerate}
      \item \ref{s:LC:s:CC} Two Clock Condition, one inside process (C1), one between process, measured by message (C2).
      \item improved and 3-D space-time diagram.
      \item \ref{s:LC:s:IR} Two Implementation Rule, one inside process (IR1), one between process , measured by message, $T_m = C_i\langle a \rangle$ (IR2).
  \end{enumerate}
  \item Ordering the Events Totally
  \begin{enumerate}
      \item \ref{s:TO:s:DTO} Definition of total ordering using Logical Clock and order among processes, Symbol $\Rightarrow$
      \item \ref{s:TO:s:ME} Solve mutual exclusion problem, fixed number of processes share a single resource.
      \item Specify the synchronization as a State Machine.
      \item Pitfall, handling failure (related paper is referred).
  \end{enumerate}
  \item Anomalous Behavior
  \begin{enumerate}
      \item Precedence information based on messages external to the system cause anomalous behavior.
      \item Solution 1, \textit{explicitly introduce info about the ordering $\rightarrow$ to the system} by user.
      \item Solution 2, Strong Clock Condition, thus use Physical Clocks to eliminate anomalous behavior.
  \end{enumerate}
  \item Physical Clocks
  % TODO: this part is a lot harder, and I skipped the Proof of the Theorem in appendix /w\
  \begin{enumerate}
      \item Requirements for the clock, individual (PC1), between clocks, small variation (PC2).
      \item Requirements for preventing anomalous behavior.
      \item Algorithm to ensure PC2, clock synchronization.
      \item The length of time for synchronizing clocks.
  \end{enumerate}
\end{enumerate}

Following is the reading notes for each section

\subsection{Introduction} \label{s:intro}

\subsubsection{Definition of Distributed Systems} \label{s:intro:s:dist}

\begin{quote}
  A distributed system consists of a collection of distinct processes which are
spatially separated, and which communicate with one another by exchanging messages. \\
  A system is distributed if the message transmission delay is not negligible compared
to the time between events in a single process.\cite{l} \\

% TODO: right align and add a horizontal line
--\ \textit{Leslie Lamport}
\end{quote}

The order of event is simple if we only consider physical time, like I woke on 8:00AM happens before
I eat my breakfast on 8:30AM.
But communication takes time due to spatial separation or other facts.

% FIXME: this example feels quite wired, sth wrong
A real world example:
I woke up on 8:00AM, send a message to Jack on 8:05AM and ate breakfast on 8:10AM,
Jack saw the message on 8:30AM.
So although in physical time breakfast \textit{happens before} Jack saw the message, they are \textit{concurrent}
in the view of the system. Jack don't know I was having breakfast on 8:10AM. %TODO: even if I tell him I will
% eat on 8:10 and he got message 8:05, he just know I will do, but don't know I really did unless I send him
% another message

\subsection{The Partial Ordering} \label{PO}

\subsubsection{Definition of Order} \label{s:PO:DO}

Order need to be observed with the system.
Physical clocks are not perfectly accurate and do not keep precise physical time\cite{l}.

\medskip

Define \textit{happens before} ($\rightarrow$) as following

\begin{enumerate}
  % TODO: may use the word in paper directly for PO1, due to C1
  \item \label{PO1} Inside a process, $\rightarrow$ is just like physical time.
  \item \label{PO2} If message is sent from process $P_i$ to $P_j$,
  $a$ is the sending of the message,  $b$ is receiving of that message, then $a \rightarrow b$.
  \item $if\ a \rightarrow b\ and\ b \rightarrow c, then\ a \rightarrow c$.
  % TODO: IMPL the counter implementation
  \item $if\ a \centernot\rightarrow b\ and\ b \centernot\rightarrow a$, then a is \textit{concurrent} with b.
\end{enumerate}

\subsection{Logical Clocks} \label{LC}

Define \textit{Clock} $C$ as following\cite{l}

\begin{itemize}
  \item just a way of assigning numbers to event, no relation with physical time, can be implemented using mere counter.
  \item $C_i$ for each process $P_i$ where a number $C_i\langle a \rangle$ is assigned to event $a$ in that process.
  \item function $C$ represents the entire system of clocks, so $C\langle b \rangle = C_i \langle b \rangle$ if $b$ is an event in process $P_i$.
\end{itemize}

\subsubsection{Clock Condition} \label{s:LC:s:CC}

Define \textit{Clock Condition} as following

\medskip

\ \ \ \ For any events $a, b$: if $a \rightarrow b$ then $C\langle a \rangle < C\langle b \rangle$

\medskip

By extending \ref{PO1} and \ref{PO2} in \ref{PO} we have

\begin{enumerate}
  \item \label{C1} C1. If $a$ and $b$ are events in process $P_i$, and $a$ comes before $b$, then $C\langle a \rangle < C\langle b \rangle$\cite{l}
  \item \label{C2} C2. If message is sent from process $a$ to $b$, then $C\langle a \rangle < C\langle b \rangle$\
\end{enumerate}

\subsubsection{Implementation Rule} \label{s:LC:s:IR}

By extending \ref{C1} and \ref{C2} we have the implementation rule

% TODO: IMPL
\begin{enumerate}
  \item \label{IR1} IR1. Each process $P_i$ increments $C_i$ between any two successive events\cite{l}.
  \item \label{IR2} IR2.
  \begin{enumerate}
    \item If event $a$ is the sending of a message $m$ by process $P_i$, then $m$ contains a timestamp $T_m=C_i\langle a \rangle$\cite{l}.
    \item Upon receiving a message $m$, process $P_j$ sets $C_j$ greater than or equal to its present value and greater than $T_m$\cite{l}.
  \end{enumerate}
\end{enumerate}

\subsection{Ordering the Event Totally} \label{TO}

\subsubsection{Definition of Total Ordering} \label{s:TO:s:DTO}

% TODO: use href here to avoid the awkward section number.
By extending partial ordering \ref{PO}, we have total ordering defined as following.

\medskip

$a \Rightarrow b$ if and only if one of the following is satisfied

\begin{enumerate}
  \item $C\langle a \rangle < C\langle b \rangle$
  % TODO: the arbitrary order part
  \item $C\langle a \rangle = C\langle b \rangle$ and $P_i \prec P_j$
\end{enumerate}

Order between processes is used to break ties and any order $\prec$ can be used\footnote{$\prec$ can by any arbitrary order, see \ref{term:arbitrary_order}}.
Thus result in different total ordering relations $\Rightarrow$ for a unique partial ordering relation $\rightarrow$.

% TODO: use subsubsection for the example, may use subsubsection for all of them

\subsubsection{Solve Mutual Exclusion} \label{s:TO:s:ME}

A fixed number of processes share a single resource. The algorithm must satisfies the following

\begin{enumerate}
  \item Unless the process using the resource release it, other process cannot be granted.
  \item Requests for the resources are fulfilled in the order they are made.
  \item If every process which is granted the resource eventually releases it, then every request is eventually granted\cite{l}.
\end{enumerate}

A simple centralize solution won't work. ie: $P_1$ send a request (a) and then a message to $P_2$ (b), $P_2$ receive the message (c) and also send a request (d).
we have $a \rightarrow b \rightarrow c \rightarrow d$, so $P_1$ should get the resource first. But it's possible $P_2$'s request arrive before $P_1$.
If the central merely grant process resource based on the order it receive requests, it can only be correct when all messages and requests are delivered in a time
shorter than time between actions inside a process.
% TODO:
TODO: will add timestamp on request and message fix this.
% > we implementation a system of clocks with rules IR1 and IR2 and use them to define a total ordering => of all events

\medskip

Assumptions to avoid implementation details\footnote{Using id, retry and ACK like in TCP can avoid those assumptions}

\begin{enumerate}
  \item Message will not be dropped.
  \item Message between two processes are received in the order they are sent
  % TODO: what if there are levels between process and block them from talking to processes in other levels.
  \item A process can send message directly to every other process
\end{enumerate}

Each process has a private message queue with initial value of $T_0:P_0$, while $P_0$ is the chosen one and
$T_0$ is less than the initial value of any clock.

Following five rules are defined.

\begin{enumerate}
  % TODO: how to obtain the timestamp
  \item $P_i$ send $T_m:P_i$ request to ALL the processes (include itself, which means put it in its own queue directly),
  where $T_m$ is the timestamp of the message obtained from the clock $C_i$.
  % TODO: not sure about the reason for ignore ack, kind of circular explanation
  \item $P_j$ put message $T_m:P_i$ in its queue when he receive it and sent an ACK to $P_i$. The ACK can be ignored if
  $P_j$ has sent $P_i$ a message $T_n:P_j$ where $T_m < T_n$. (Because the clock only ticks when there is event)
  % TODO: will there be multiple Tm:Pi in request queue
  % TODO: what's the timestamp for the release, Tm or the time of release.
  \item To release the resource, process $P_i$ removes any $T_m:P_i$ requests in its queue and send a (timestamped) $P_i$
  release to every other process\cite{l}.
  \item When $P_j$ receive a $P_i$ release message, it removes any $T_m:P_i$ request from its queue.
  % TODO: one of  OR all, it's all
  \item Resource is granted to $P_i$ when all the following conditions are satisfied
  \begin{enumerate}
    % It's time for me
    \item There is a $T_m:P_i$ request in its queue, with $T_m$ $\Rightarrow$ all the other requests' timestamps.
    % Know all the other process
    \item $P_i$ has received from ALL the other processes a message with timestamp later than $T_m$
  \end{enumerate}
\end{enumerate}

TBD: state machine \\
TBD: Pitfalll and tolerant

\subsection{Anomalous Behavior}

TBD

\subsection{Physical Clocks}

TBD

\section{Simple implementation}

TBD \\
Example: \url{https://neerajkumar.wikispaces.com/file/view/DS+LAB+MANUAL_ECS701.pdf}

% Some simple C code could do that
%
% \begin{lstlisting}[language=C]
% #include <stdio>
%
% int main(char* args){
%   print("Hello Clock!")
% }
% \end{lstlisting}

\section{Supplemental Materials}

\subsection{Video and Slides}

\begin{itemize}
  \item Udacity \url{https://www.youtube.com/watch?v=z2oY8LWHYq8}
  \item Papers we love: Time clocks and the reordering of events PWL SF 14-Jul-2016 Video: \url{https://www.youtube.com/watch?v=CWF3QnfihL4\&t=1953s}
  \item Papers we love: Time clocks and the reordering of events PWL SF 14-Jul-2016 Slide: \url{https://speakerdeck.com/pborrill/time-clocks-and-the-reordering-of-events-pwl-san-francisco-14-jul-2016}
\end{itemize}

More can be found on https://github.com/at15/CMPS232/issues/6, thanks to @czheo.

\subsection{Terms} \label{l1}

Since most terms I learned about math and computer science was in Chinese. I found a lot of terms blocking
me from understanding this paper, so I listed them below for fellow students whose mother language is not English.

\medskip

% TODO: may use subsubsection

\subsubsection{Converse, Inverse, Contrapositive}

Given a statement $if\ p, then\ q$, $p$ is called hypothesis, $q$ is called $conclusion$.\footnote{\url{http://hotmath.com/hotmath_help/topics/converse-inverse-contrapositive.html}}.

\begin{itemize}
    \item Converse  $if\ q, then\ p$
    \item Inverse $if\ not\ p, then\ not\ q$
    \item Contrapositive $if\ not q, then\ not\ p$
\end{itemize}

If the statement is true, then the contrapositive is also true.
And the inverse is the contrapositive for the converse, so if the inverse is true, the converse is true.

\subsubsection{Mutual Exclusion}

% FIXME: this website has annoying ads, why its the first result in google
Mutual Exclusion\footnote{\url{https://www.techopedia.com/definition/25629/mutual-exclusion-mutex}} is also known as mutex,
which you may have heard when reading books about multithread.
It is \textit{a program object that prevents simultaneous access to a shared resource}.

\subsubsection{Arbitrary order} \label{term:arbitrary_order}

An order that does not have a standard. You can order it by the way you want as long as you a order at last.
It is mentioned in \ref{TO} for ordering processes in total ordering.

\section{Discussion}

% What problem does the paper solve? Is is important?
% How does it solve the problem?
% What alternative approaches are there? Are they adequately discussed in the reading?
% How does this work relate to other research, whether covered in this course or not?
% What specific research questions, if any, does the paper raise for you?

\subsection{Problem solved}

Determine order partially and totally. % TODO: why is this important

The paper solve the problem of total ordering in distributed system.
It is important, because it can solve many synchronization problems.

By introducing logical clocks and a method to adjust physical clocks.

\subsection{Alternative}

No, no alternative for logical clocks. Oops I guess there are, things
like vector clock and stuff may work LOL. Saw them from blog posts when googling.

\section{Related research}

Too many related researches

\section{Q\&A}

% TODO: got this question when writing the introduction part
Q: When deal with message, the 'time' of send and receive message is regarded as same timestamp? \\
A: The clock must ticks between the message,  % TODO: find the pos in paper

\medskip

Q: How visualize the space-time diagram to 3D \\
A: IDK

\medskip

Q: Logic clock ticks based on event, but it's kind of like the question of chicken and eggs, so where the timestamp comes from.
If we use the counter, do we need a central counter, or each of them just use their own and then adjust. \\
A: IDK

\medskip

Q: What if the processes number changes dynamically, some quit some join. The paper mentioned the least time required for all the
clocks in the system to synchronize. Can it be extended to the above situation. \\
A: IDK

\medskip

Q: The example in this paper use a lot of assumptions to simplify the proof.
In real world applications, how engineers deal with these assumptions like messages
arrive in order. \\
A: IDK

\section{Changelog}

\begin{itemize}
  \item 09/25/2016 15:17 \ Adjust layout and section order
  \item 09/25/2016 14:03 \ Finish most part of paper walk through.
  \item 09/24/2016 00:00 \ Initial version.
\end{itemize}

\printbibliography

\end{document}

This is never printed
