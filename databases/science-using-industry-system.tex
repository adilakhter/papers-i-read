\documentclass[12pt,a4paper,oneside]{article}
\usepackage[a4paper]{geometry}
\setlength\parindent{0pt}
\usepackage{hyperref}

\begin{document}

\begin{itemize}
  \item Name: Pinglei Guo
  \item Assignment: Rethinking Data-Intensive Science Using Scalable Analytics Systems
  \item Last Modified: \today
\end{itemize}

1. What is the problem authors are trying to solve

\medskip

Current analytics system for scientific worloads are low efficient and error prone

\bigskip

2. How does the authors’ approach or solution improve on previous approaches to that problem

\medskip

\begin{itemize}
  \item Utilizes existing industry solutions, Spark, Parquet
  \item More general by using a OSI like layer model
  \item Require less custom logic by the user
\end{itemize}

\bigskip

3. Why is this work important

\medskip

Because science data are getting bigger, and using HPC may no longer be the ultimate solution,
cloud service makes running large scale analysis cheaper

\bigskip

4. Your comments/questions

\medskip

Questions

\begin{itemize}
  \item Providing a general system is good, but I think for some specific areas, it may not be a bad idea to build a specific system, like for genomic,
I could build a system that works for genomic but not work for astronomy. I think pharma companies can make huge profit from genomic researches, and I
don't think they would like to sacrifice performance for other scientific fields.
  \item This paper reminds me of SETI https://setiathome.berkeley.edu/ , which I read from magzine when I didn't have internet access. I think this could
be another future for large scale scientific computing, no matter how large AWS is, it can not beat the world. Pepole around world were generating bitcoin,
whose only purpose is been a currency. What if the computation can also benifit scientific computing
  \item Combing a bunch of frameworks and make them work is a hard work, and I think that's why papers like this and companies like Cloudera, MapR, Datastax exists.
However I believe in the future, paper like this will become legacy, so does those companies who provide consulating and enterprise version for open source softwares.
The advancement in AI will outperform most huamn when it comes to combine and tuning stuff. i.e. Peloton, the self driving database.
  \item I think Prof. Carlos's group had published some paper called Sci Hadoop, but never looked into them.
\end{itemize}

\end{document}
