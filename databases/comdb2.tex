\documentclass[12pt,a4paper,oneside]{article}
\usepackage[a4paper]{geometry}
\setlength\parindent{0pt}
% \usepackage{cancel}
\usepackage[normalem]{ulem} % either use this (simple) or

\begin{document}

\begin{itemize}
  \item Name: Pinglei Guo
  \item Assignment: Comdb2 Bloomberg's Highly Available Relational Database System
  \item Last Modified: \today
\end{itemize}

1. What is the problem authors are trying to solve

\medskip

Build a tfull featured RDBMS that can scale, no compromise for HA.

\bigskip

2. How does the authors’ approach or solution improve on previous approaches to that problem

\medskip

\begin{itemize}
  \item Didn't settle for eventual consistency
  \item Full transcation support, not just micro transaction
  \item Support rich stored procedures using a dialect of lua
\end{itemize}

\bigskip

3. Why is this work important

\medskip

\begin{itemize}
  \item It's another example (among the many like spanner) showing that the loose schema and transaction advocated by NoSQL
  is not the only solution to database that need to scale and have full features.
  \item It has been used in BLP since 2004, and we didn't hear bad names from BLP due to its database.
\end{itemize}

\bigskip

4. Your comments/questions

\medskip

Questions

\begin{itemize}
  \item \sout{It is 2017 and still no sign of its code on GitHub}.
  \item It is now https://github.com/bloomberg/comdb2
  \item Speical hardware is needed just like spanner to acheive HA and consistency in one system.
  One good thing about real world compared with theory, what it is impossible in theory (i.e. CAP) can be achieved in real world as long as you have some tolerance.
  \item Percona, the variant of MySQL is also mentioned in the evaluation, I think it's a better choice for most medium sized companies,
  for they don't have the experience and resource to build a new database like BLP and SAP did, using databases with commercial support
  is the best way to keep growing their bussiness.
  \item the complex logic has to be somewhere in the system, Comdb2 takes the burden into implementaion and API, giving the application
  developer more time to focus on the bussiness logic. But one could argue that if the application is strong enough, the developer could
  actually spend time on developing higher level things, like payment as a service free developer from keeping track of orders and payments,
  they don't even need to know what database their PaaS provider is using.
\end{itemize}

Miscellaneous

\begin{itemize}
  \item The author of SQLite (Richard Hipp) is mentioned in conclusion, though what he did is not mentioned.
\end{itemize}

\end{document}
